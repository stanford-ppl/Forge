%%%%%%%%%%%%%%%%%%%%%%%%%%%%%%%%%%%%%%%%%
% Short Sectioned Assignment
% LaTeX Template
% Version 1.0 (5/5/12)
%
% This template has been downloaded from:
% http://www.LaTeXTemplates.com
%
% Original author:
% Frits Wenneker (http://www.howtotex.com)
%
% License:
% CC BY-NC-SA 3.0 (http://creativecommons.org/licenses/by-nc-sa/3.0/)
%
%%%%%%%%%%%%%%%%%%%%%%%%%%%%%%%%%%%%%%%%%

%----------------------------------------------------------------------------------------
%	PACKAGES AND OTHER DOCUMENT CONFIGURATIONS
%----------------------------------------------------------------------------------------

\documentclass[paper=letter, fontsize=11pt]{scrartcl} % Letter paper and 11pt font size

\usepackage[T1]{fontenc} % Use 8-bit encoding that has 256 glyphs
\usepackage{fourier} % Use the Adobe Utopia font for the document - comment this line to return to the LaTeX default
\usepackage[english]{babel} % English language/hyphenation
\usepackage{amsmath,amsfonts,amsthm} % Math packages

\usepackage{float}		% floating objects
\usepackage{graphicx}
\usepackage{caption}
\usepackage{subcaption}

%\usepackage{lipsum} % Used for inserting dummy 'Lorem ipsum' text into the template
\usepackage{sectsty} % Allows customizing section commands
\allsectionsfont{\centering \normalfont\scshape} % Make all sections centered, the default font and small caps
\usepackage{fancyhdr} % Custom headers and footers
\usepackage{hyperref}
\usepackage{listings}
\usepackage{color}
\usepackage{dirtree}

\pagestyle{fancyplain} % Makes all pages in the document conform to the custom headers and footers
\fancyhead[L]{} % No page header - if you want one, create it in the same way as the footers below
\fancyhead[C]{} % No page header - if you want one, create it in the same way as the footers below
\fancyhead[R]{Fall 2015-16} % No page header - if you want one, create it in the same way as the footers below
\fancyfoot[L]{} % Empty left footer
\fancyfoot[C]{} % Empty center footer
\fancyfoot[R]{\thepage} % Page numbering for right footer
\renewcommand{\headrulewidth}{0pt} % Remove header underlines
\renewcommand{\footrulewidth}{0pt} % Remove footer underlines
\setlength{\headheight}{13.6pt} % Customize the height of the header

\numberwithin{equation}{section} % Number equations within sections (i.e. 1.1, 1.2, 2.1, 2.2 instead of 1, 2, 3, 4)
\numberwithin{figure}{section} % Number figures within sections (i.e. 1.1, 1.2, 2.1, 2.2 instead of 1, 2, 3, 4)
\numberwithin{table}{section} % Number tables within sections (i.e. 1.1, 1.2, 2.1, 2.2 instead of 1, 2, 3, 4)

\setlength\parindent{0pt} % Removes all indentation from paragraphs - comment this line for an assignment with lots of text
\setlength\parskip{0.5em}

\definecolor{codegreen}{rgb}{0,0.6,0}
\definecolor{codeblue}{rgb}{0,0,0.6}
\definecolor{codered}{rgb}{0.6,0,0}
\definecolor{codegray}{rgb}{0.5,0.5,0.5}
\definecolor{codepurple}{rgb}{0.58,0,0.82}
\definecolor{backcolour}{rgb}{0.95,0.95,0.92}

% "define" Scala
\lstdefinelanguage{scala}{
  morekeywords={abstract,case,catch,class,def,%
    do,else,extends,false,final,finally,%
    for,if,implicit,import,match,mixin,%
    new,null,object,override,package,%
    private,protected,requires,return,sealed,%
    super,this,throw,trait,true,try,%
    type,val,var,while,with,yield},
  otherkeywords={=>,<-,<\%,<:,>:,\#,@},
  sensitive=true,
  morecomment=[l]{//},
  morecomment=[n]{/*}{*/},
  morestring=[b]",
  morestring=[b]',
  morestring=[b]"""
}

\lstdefinestyle{mystyle}{
    backgroundcolor=\color{backcolour},
    commentstyle=\color{codeblue},
    keywordstyle=\color{codegreen},
    numberstyle=\tiny\color{codegray},
    stringstyle=\color{codered},
    basicstyle=\footnotesize,
    breakatwhitespace=false,
    breaklines=true,
    captionpos=b,
    keepspaces=true,
    numbers=left,
    numbersep=5pt,
    showspaces=false,
    showstringspaces=false,
    showtabs=false,
    tabsize=2
}

\lstset{style=mystyle}
%----------------------------------------------------------------------------------------
%	TITLE SECTION
%----------------------------------------------------------------------------------------

\newcommand{\horrule}[1]{\rule{\linewidth}{#1}} % Create horizontal rule command with 1 argument of height

\title{
\normalfont \normalsize
\textsc{Pervasive Parallelism Lab\\Stanford University} \\ [25pt] % Your university, school and/or department name(s)
\horrule{0.5pt} \\[0.4cm] % Thin top horizontal rule
\huge Delite Hardware Definition Language\\ % The assignment title
\horrule{2pt} \\[0.5cm] % Thick bottom horizontal rule
}

% \date{\normalsize\today} % Today's date or a custom date
\date{} % Today's date or a custom date
\author{Raghu Prabhakar} % Not using authors for this one


\begin{document}

\maketitle % Print the title

\section{Introduction}
This document is meant a technical specification of the Delite Hardware
Definition Language, or DHDL. DHDL is a domain-specific language designed to
serve as a low-level compiler intermediate representation (IR) to generate efficient
hardware accelerators. DHDL is embedded in Scala, so a DHDL program is really a Scala
program with a few special imports. An application written in DHDL describes a \emph{hierarchical data flow
graph} which consists of many kinds of \emph{nodes}. The graph can be used for further
analysis or low-level code generation.

DHDL is a research project under active development, so this document will frequently get obsolete.
Every effort will be made to keep the contents of this document consistent with the actual
implementation, but the author makes no guarantees.

\section{Setup and Installation}
\subsection{System Requirements}
DHDL requires that the following tools be installed:
\begin{enumerate}
  \item \texttt{sbt} - The Scala build tool. You can download and install it from here: \href{http://www.scala-sbt.org/}{http://www.scala-sbt.org/}
  \item \texttt{git}
\end{enumerate}
\subsection{Installation}
Clone the source code, build sources using \texttt{make}.
\begin{lstlisting}[language=sh, caption=Build and install DHDL]
git clone https://raghup17@bitbucket.org/raghup17/dhdl.git
cd dhdl
make
\end{lstlisting}

\begin{lstlisting}[language=sh, caption=Makefile targets and usage]
  # Builds all DHDL sources and applications
  make

  # Builds this manual from source, generates scaladoc and creates a symlink to the
  # generated documentatio in the docs folder for easy access
  make doc

  # Deletes generated files, documents and compiled classes
  make clean

  # make clean + removes Tex-generated files for manual
  make distclean

  # Run application APP with arguments a1 a2
  bin/dhdl APP a1 a2
\end{lstlisting}
The \texttt{Makefile} defines a set of useful targets to build and run applications.
Listing \ref{lst-make-targets} describes the targets.


\subsection{Directory Structure}
\renewcommand*\DTstylecomment{\rmfamily\color{codegreen}\textsc}
\renewcommand*\DTstyle{\ttfamily\textcolor{codeblue}}
\dirtree {%
.1 dhdl.
.2 apps \DTcomment{DHDL applications folder. New applications must go here}.
.2 docs.
.3 manual \DTcomment{Source for this manual}.
.3 api \DTcomment{Symlink to scaladoc documentation}.
.2 project \DTcomment{DHDL project definition and build rules}.
.2 src.
.3 codegen.
.4 dot \DTcomment{GraphViz code generator package}.
.4 maxj \DTcomment{MaxJ code generator package}.
.3 graph \DTcomment{Package defining all DHDL nodes and types}.
.3 main \DTcomment{Wrapper trait that must be mixed in with every DHDL application}.
.2 test \DTcomment{Tests - empty for now}.
}

\section{Hello, DHDL!}
First program listing \\
Build, run \\
Anatomy of a DHDL program \\


\section{Node Hierarchy}
Node hierarchy tree \\
Describe each node and its parameters \\
Describe parent relationship \\

\section{Types}
Describe FixPt, FloatPt objects \\
Describe how to use them in DHDL program \\
Describe type errors on incorrect usage

\section{Writing an analysis pass}

Describe API \\
Describe simple pass \\


\section{Writing a backend}

Describe that codegen is an analysis pass that produces files \\
Currently we have MaxJ and GraphViz codegen \\
Describe a simple code generator

\section{Additional Resources}

DHDL: \\
Pointer to Doxygen \\
My email address \\

\end{document}
